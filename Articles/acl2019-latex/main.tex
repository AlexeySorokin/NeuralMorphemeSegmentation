%
% File acl2019.tex
%
%% Based on the style files for ACL 2018, NAACL 2018/19, which were
%% Based on the style files for ACL-2015, with some improvements
%%  taken from the NAACL-2016 style
%% Based on the style files for ACL-2014, which were, in turn,
%% based on ACL-2013, ACL-2012, ACL-2011, ACL-2010, ACL-IJCNLP-2009,
%% EACL-2009, IJCNLP-2008...
%% Based on the style files for EACL 2006 by 
%%e.agirre@ehu.es or Sergi.Balari@uab.es
%% and that of ACL 08 by Joakim Nivre and Noah Smith

\documentclass[11pt,a4paper]{article}
\usepackage[hyperref]{acl2019}
\usepackage{times}
\usepackage{latexsym}

\usepackage{url}

%\aclfinalcopy % Uncomment this line for the final submission
%\def\aclpaperid{***} %  Enter the acl Paper ID here

%\setlength\titlebox{5cm}
% You can expand the titlebox if you need extra space
% to show all the authors. Please do not make the titlebox
% smaller than 5cm (the original size); we will check this
% in the camera-ready version and ask you to change it back.

\newcommand\BibTeX{B\textsc{ib}\TeX}

\title{Convolutional network: baseline of state-of-the-art quality for low-resource morpheme segmentation.}

\author{Intentionally anonymous}

\date{}

\begin{document}
\maketitle
\begin{abstract}
  We apply convolutional neural networks to the task of shallow morpheme segmentation using low-resource datasets for 5 different languages. We show that both in fully supervised and semi-supervised settings our model beats previous state-of-the-art approaches. We argue that convolutional neural networks reflect local nature of morpheme segmentation better than other semi-supervised approaches.
\end{abstract}

Morpheme segmentation consists in dividing a given word to meaningful individual units, morphemes. For example, a word \textit{unexpectedly} could be segmented as \textit{un-expect-ed-ly}. The generated segmentation may be used as input representation for machine translation \cite{Mager2018b} or morphological tagging \cite{Matteson2018} or for automatic annotation of digital linguistic resources. 


\end{document}
